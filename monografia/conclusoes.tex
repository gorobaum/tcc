\section{Conclusões}
Esse trabalho mostrou que o principais fatores que influenciam no desempenho de linguagens para GPU são:
A abstração que a linguagem faz da GPU. Quanto mais rica ela for para expressar a execução de um kernel, mais
controle o programador vai obter. Tanto o controle da execução quanto o conhecimento do hardware no qual o kernel vai
executar são conhecimentos fundamentais para um programador GPGPU. Com os dois, é possível atingir o máximo
desempenho, pois o programador está o mais próximo do hardware possível.

Outro fator de grande importância é a comunicação da linguagem com a GPU. O hardware das GPUs é constantemente
melhorado, gerando arquiteturas diferentes; e novas funcionalidades são implementadas em cada atualização dentro 
de uma mesma arquitetura. Isso faz com que os drivers, o responsável por controlar o acesso do hardware pelo software,
esteja sempre atualizado para garantir que as funcionalidades mais atuais possam ser usadas.

Mesmo que tanto o OpenCL como o CUDA tenham uma ótima abstração das GPUs, o CUDA leva a vantagem em desempenho em GPUs NVIDIA 
por ter seus drivers matidos sempre atualizados.
