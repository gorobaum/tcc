\section{Conceitos e Técnologias}
\subsection{High-Performance Computability}
HPC nasceu da necessidade de poder computacional para resolver uma série de problemas, entre eles:
\begin{itemize}
  \item Previsão climática
  \item Modelação molecular
  \item Simulações físicas
  \item Física quântica
\end{itemize}
Os supercomputadores foram criados para rodar as aplicações que executavam esses objetivos. Até o final
dos anos 90 todos os supercomputadores tinham como base CPUs. Só no final da década seguinte, com o aumento
do desempenho das GPUs, que alguns supercomputadores começaram a usar GPUs como seus processadores
\subsection{GPGPU}
\subsubsection{História}
\subsubsection{Placas NVidia}
\subsection{CUDA}
\subsubsection{Modelo de Memória}
\subsubsection{Modelo de Execução}
\subsubsection{Modelo de Plataforma}
\subsubsection{Modelo de Programação}
\subsection{OpenCL}
\subsubsection{Modelo de Memória}
\subsubsection{Modelo de Execução}
\subsubsection{Modelo de Plataforma}
\subsubsection{Modelo de Programação}
