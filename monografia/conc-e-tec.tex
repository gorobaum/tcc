\section{Conceitos e Técnologias}
\subsection{High-Performance Computability}
HPC nasceu da necessidade de poder computacional para resolver uma série de problemas, entre eles:
\begin{itemize}
  \item Previsão climática
  \item Modelação molecular
  \item Simulações físicas
  \item Física quântica
\end{itemize}
Até o final dos anos 90 todos os supercomputadores tinham como base processadores vetoriais. Só no final da década seguinte, 
com o aumento do desempenho das GPUs, que alguns supercomputadores começaram a usar GPUs como suas fontes e processamento.
\subsection{GPGPU}
As GPUs modernas são compostas de centenas ou até milhares de processadores mais lentos que um processador de uma CPU,
mas quando usados em conjunto em operações paralelas chegam a ser 10 vezes mais rápidas que uma CPU. A escolha para essa
arquitetura é o modelo de execução em pipeline que as GPUs implementam. Todo dado passava por um sequencia de operações
pré-definidas dentro da GPU. Com o passar dos anos, essas operações se tornaram programaveis, primeiramente atraves
dos shaders, e depois atraves de linguagens como as descritas logo abaixo. Esse paralelismo inerente das GPUs criou
um ambiente favoravel para aplicações paralelizaveis de alta performance. Vamos estudar um pouco mais a fundo a 
arquitetura COLOCAR ARQUITTURA DA GTX260 AQUI, pois quanto mais sabemos do hardware que executará nosso programa
mais podemos otimizá-lo.
 
As GPUs com esssa arquitetura são compostas de multiprocessadores, cada um com 32 processadores. Cada multiprocessador
contém um único registrador de instruções para todos seus processadores, fazendo com que todos eles executem a mesma 
instrução, mas em dados diferentes. Isso impõe algumas complicações na escrita de programas. Temos um desempenho
maior em programas que distribuem a memória de modo que threads rodando no mesmo multiprocessador utilizem posições
que estão no mesmo bloco da memória, evitando que o gerenciador de memória tenha que fazer leituras na memória principal.

Um dos problemas com GPGPU é a memória da GPU. Ela é bem mais limitada que a acessivel pela CPU. GPUs tem, em média, 1GB
de memória, enquanto CPUs tem 4GB. Outro fator é a transferencia da memória principal do computador para a memória principal
da GPU. A transmissão é feita por um barramento PCI Express na maioria dos casos, com velocidades de até 16GB/s, dado que o
barramento esteja sendo usado somente para isso, o que é pouco provável. Essa transmissão é a parte mais lenta de todo o
processo de execução na GPU. Em alguns casos é mais viável fazer algumas operações de baixar eficiência numa GPU do que
retornar os dados para a CPU e passá-los de volta para a GPU.

\subsubsection{Stream processing}
Esse paradigma cria um ambiente que simplifica tanto o software quanto o hardware, mas limita a quantidade de problemas 
paralelizáveis que odem ser resolvídos nesse ambiente.

Stream processing se baseia em vários processadores, cada um com um pequeno cache de acesso próprio, 
rodando um mesmo conjunto de instruções ( kernels ). Cada processador irá aplicar o mesmo kernel em um
conjunto diferente de dados, e todos eles executarão seus kernels em paralelo. No final, todo o resultado é
combinado, assim resolvendo o problema inicial.

Vamos usar de exemplo a GPU usada para os testes desse trabalho, uma GeForce GTX 260. Ela contém
192 processadores, separados em blocos de 8 processadores, cada um com um cache de 16Kb. Cada bloco
tem um cache próprio, e a placa tem 896Mb de memória.

É fácil ver como esse paradigma é eficiente para resolver integrais ou multiplicações de matrizes, onde não há
dependencia de dados entre instâncias dos kernels e os dados podem ser fácilmente segmentados para caches diferentes.
Mas esse paradigma pode não ser o mais eficiente para resolver um problema clássico de programação paralela, o dos
Filósofos Famintos, onde todos os processos dependem de mais dois processos. O problema dessa depêndencia é a 
compartilhação de memória entre os dependentes. Ao modificar um pedaço de memória compartilhada, todos os processos 
que utilizam essa memória devem parar sua execução e fazer a releitura do cache, que deve ser feita de maneira sequencial.

Conhecer tanto a arquitetura como a capacidade de processamento e memória do hardware em que vamos programar é 
importante para conseguir o máximo de desempenho possível. É importante também modificar o problema em questão 
para adequá-lo a esse paradigma.

Para obter o máximo desempenho do programa, é preciso analisar como a linguagem escolhida abstrai o hardware da GPU.

\subsection{CUDA}
Compute Unified Device Architecture (CUDA) é uma arquitetura de programação para GPUs criada pela NVidia.
A versão 1.0 foi disponibilizada no inicio de 2007. Atualmente só existe um compilador para CUDA, o nvcc,
e ele só da suporte para GPUs NVidia.

O CUDA implementa um conjunto virtual de instruções e memória, tornando os programas retroativos. O compilador
primeiro compila o código em C para um intermediário, chamado de PTX, que depois será convertido em linguagem
de máquina. Na conversão do PTX para linguagem de máquina o compilador verifica quais instruções o hardware
suporte e converte o código para usar as instruções corretas. Novamente, para obter o maior desempenho possível,
é importante saber para qual versão o código final será compilado, pois na passagem do código de uma versão
maior para uma menor não existe a garantia que o algoritmo seguira as mesmas instruções, o compilador pode
mudar um conjunto de instruções para outro menos eficiênte.

A inicialização dos recursos que o CUDA necessita para a comunicação com a GPU é feita no background da
aplicação no momento da primeira chamada de alguma das diretivas do CUDA. Essa primeira diretiva terá um
tempo maior de execução que chamadas subsequentes a mesma diretiva.
\subsubsection{Modelo de Memória}

\subsubsection{Modelo de Execução}
\subsubsection{Modelo de Plataforma}
\subsubsection{Modelo de Programação}
\subsection{OpenCL}
\subsubsection{Modelo de Memória}
\subsubsection{Modelo de Execução}
\subsubsection{Modelo de Plataforma}
\subsubsection{Modelo de Programação}
