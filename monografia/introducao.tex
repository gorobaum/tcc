\section{Introdução}
\subsection{Motivação}
Em Computação de alto desempenho (HPC) existe uma parcela de supercomputadores montados com base em placas
de processamento gráfico (GPU). O termo GPGPU (General-purpose computing on graphics processing units) é
\\usado para denotar o uso de GPUs como a principal unidade de computação em programas de mais amplo espectro.

Duas linguagens são muito utilizadas atualmente para programação em ambientes GPGPU, o OpenCL (Open Computing Language) 
e o CUDA (Compute Unified Device Architecture). O OpenCL é reconhecido por executar em ambientes GPGPU com processadores genéricos, 
enquanto o CUDA é construido envolta de ambientes NVIDIA.

\subsection{Objetivos}
O objetivo do estudo é comparar a eficiência de programas escritos nessas duas linguagens rodando em uma placa NVIDIA GeForce GTX 460
e comparar o modo com que eles abstraem os recursos de uma GPU, tornando possível executar programas genéricos na mesma.

\subsection{Problemas a serem resolvidos}
Para realizar a comparação de eficiência entre as linguagens é necessário desenvolver dois algoritmos para testes, um que verifique
a capacidade de processamento da linguagem e outro a capacidade de manipular memória. Além disso, é necessário comparar as abstrações
para entender de onde vem a diferença de desempenho entre as linguagens.
