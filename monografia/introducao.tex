\section{Introdução}
\subsection{Motivação}
Em Computação de Alto Desempenho (HPC) existe uma parcela de supercomputadores montados com base em placas
de processamento gráfico (GPU). O termo GPGPU (General-purpose computing on graphics processing units) é
usado para denotar o uso de GPUs para executar programas de mais amplo espectro.

Duas linguagens são muito utilizadas atualmente para programação em ambientes GPGPU, OpenCL (Open Computing Language) 
e CUDA (Compute Unified Device Architecture).

\subsection{Objetivos}
O objetivo do estudo é comparar a eficiência de programas escritos nessas duas linguagens rodando em uma placa NVidia GeForce GTX 260.
\subsection{Problemas a serem resolvidos}
Para realizar essa comparação de eficiência, devemos entender como as linguagens funcionam, suas semelhanças e
diferenças, e criar um método que seja justo para comparar programas semelhantes escritos nas duas linguagens.
